\documentclass{article}

\usepackage{hyperref}
\usepackage[english]{babel}
\usepackage[
    backend=biber,
    style=numeric
]{biblatex}

\addbibresource{AdaLovelace.bib}

\begin{document}

\begin{titlepage}
    \centering
    \begin{center}
        {\Large Ministerul Educației Naționale}\\
        {\Large Colegiul Național Vlaicu Vodă}\\
        {\Large Curtea de Argeș}\\
        \vfill
        {\Large Lucrare de atestat}\\
        \vspace*{0.8cm}
        {\Large \textbf{Ada Lovelace}}\\
        \vfill
        {\Large Candidat: Ică Alexandru Gabriel}\\
        \vspace*{0.8cm}
        {\Large Profesor coordonator: Cantu Daniela}\\
        \vspace*{0.8cm}
        {\Large An școlar 2018-2019}
    \end{center}
\end{titlepage}

\pagenumbering{arabic}

\tableofcontents
\newpage

\section{Why Ada Lovelace}

Computers play a substantial role in our modern society. Phones, laptops, the internet are all part of our lives. Modern technology is so rapidly evovling that people have a hard time getting used to it. At the same time technology makes some task much more convenient, think about ordering something online. How would that work before the digital age? The computing industry is probably the most influential one that ever existed. I sometimes like to think about how all of this could be achieved in such a short period of time(the World Wide Web was standardized in 1989 \cite{TheInternet}).

\vspace{0.3cm}

Back in the sixth grade, at the very beginning of the scholar year I was introduced a class of information technology. At that time anything related to computers was completely foreign to me. I was skeptical about the subject, thinking I would not be interested in it since it. I could not have been more wrong. After the first class of the subject I found out it was very interesting and I wanted to know more about IT. It did not take very long untill I was completely fascinated by the subject and I knew this was exactly what I wanted to do.

\vspace{0.3cm}

Since I am interested in computers and I want to become a computer scientist it is natural that I am attracted by computer science related things. I am also the kind of human that wants to know how stuff works behind the scenes. Given how much abstraction there is today in even the most basic tasks(like calling somebody on the phone or paying with a credit card) I think fewer and fewer people understand how much effort has been put in transforming this world to what it is right now and trying to comprehend that effort is a very valuable attribute.

\vspace{0.3cm}

What better way to start than the very beginning of computing, when things were much simpler, and seeing how things progressed from there. Unfortunately, presenting the entire evolution of computing would be way too much, but I think just the beginning is enough to get the reader interested in the topic and that is exactly why I chose Ada Lovelace, a very smart woman that I am going to introduce to you next.

\newpage

\section{Introduction}

Many people know that the first human to be considered a programmer, also known as "somebody who knows things about computers", was a women. However, many do not know that her name was Ada Lovelace and that she was actually a mathematician \cite{Wikipedia}.

\vspace{0.3cm}

Back in her day(we are talking mid 1800s) there was no such thing as a computer as we know it today, though there was a thing called the "Analytical Engine" which was Charles Babbage's proposed mechanical general-purpose computation machine, in 1837. It used things called punch cards to operate and looked more like an engine than what we consider a computer, but it was actually very capable for its time and is considered the first Turing-complete computer(which means that it can compute anything, although in cumbersome ways). At first Babbage used it for pure calculation like square root, sin or logarithm, but Ada was the first one to recognize that the machine had applications way beyond this.

\vspace{0.3cm}

She published the first algorithm for Babbage's computer which calculated Bernoulli numbers, which is considered the first computer algorithm \cite{MentalFloss}. She was an unquestionable visionary: she understood that numbers could be used to represent more than just quantities and that the machine could be used to produce graphics, compose music and help scientists visualize their data - all of which came true in another hundred years.

\vspace{0.3cm}

The fact that she could understand the computer of Babbage so well, to the point where she took the invention of its creator to another level, combined with her vision of what a computer could be used for while living in an era when such a thing was very new, all while being in her early 20s makes her a very interesting person for me, and that's why I chose to find out more things about this woman.

\vspace{0.3cm}

The thing that sets her apart is her mindset. The very first idea of something that could automate the boring task of number crunching came in the late 1700s when economists were getting bored of doing calculations over and over again. Babbage was the first to come up with something exactly for this purpose, but he only focused on that capability. Ada was thinking ahead and saw that computers could automate a lot more tasks. Considering computer science is probably one of the most in-demand skills today, with technology being so important in the infrastructure of this world and the interest in this domain in the future only increasing, you can see how, suprisingly, a woman forever changed the computing industry, thus indirectly changing how the entire world operates.

\newpage

\section{Early life}

Ada Lovelace was born on December 10, 1815 \cite{famousscientists}, in London, first named Augusta Ada Byron(surname was changed after she got married). She came from a family of intelectuals: her father was a well knowm poet, Lord Byron, and her mother, Lady Byron, was very enthusiastic about mathematics and the sciences. Sadly, the personality of her father was very unstable and he abandoned her and her mother when Ada was only one month old and he died when she was eight years old and she did not get to know her father. Her mother did not have a close relationship with her daughter either, as Ada was often left in the care of her maternal grandmother.

\vspace{0.3cm}

Ada was taught mathematics and science, two very unusual subjects for an aristocratic girl in the mid-1800s, at her mother's insistence. Lady Byron taught that this way she would not gain her father's instability. Ada was also forced to lie for extended periods of time because her mother believed it would help her develop self-control \cite{biographydotcom}.

\vspace{0.3cm}

During her childhood she was very often ill. At the age of eight she developed headaches that obscured her vision and in 1831 she was walking with the help of crutches after being paralysed after a bout of measles. Despite this, her mathematical and technological skills continued to improve. At the age of twelve she decided that she wanted to fly. To achieve this, she started investigating different materials and sizes to prepare the wings. She went as far as examining the anatomy of birds to find out the perfect proportion between the wings and the body.

\vspace{0.3cm}

The most noticeable event in her early life was most definitely meeting Charles Babbage at the age of 17 which happened with the help of her tutor, Mary Somerville, with whom Ada became very close friends. Being an aristocrat, she had other acquaintances, including scientists Andrew Crosse, Michael Faraday, Sir David Brewster and the author Charles Dickens. By this age, her mathematical abilities began to emerge and interest in mathematics dominated most of her adult life. She often questioned basic assumptions by integrating poetry and and science. While studying differential calculus, she wrote: \cite{Wikipedia}
%\vspace*{\fill}
\vspace{0.3cm}
\begin{quote}
    \textit{
        I may remark that the curious transformations many formulae can undergo, the unsuspected and to a beginner apparently impossible identity of forms exceedingly dissimilar at first sight, is I think one of the chief difficulties in the early part of mathematical studies. I am often reminded of certain sprites and fairies one reads of, who are at one's elbows in one shape now, and the next minute in a form most dissimilar[...]
}
\end{quote}
\vspace*{\fill}

Lovelace believed that intuition and imagination were critical parts to effectively applying mathematical and scientific concepts. She valued metaphysics as much as mathematics, viewing both as tools for exploring "the unseen worlds around us" \cite{Toole}.

\newpage

\section{Work}

\subsection{First computer Program}

\vspace*{\fill}
\begin{quote}
    \textit{
        The Analytical Engine weaves algebraic patterns just as the Jacquard loom weaves flowers and leaves.
    }
    \cite{AdaLovelaceBabbageEngine}
\end{quote}
\vspace*{\fill}

Lovelace was always interested in scientific developments that were popular during the day, for example phrenology \cite{Phrenology} and mesmerism \cite{Mesmerism}. In 1844 she told Woronzow Greig about "a calculus of the nervous system" \cite{CalculusOfNervousSystem}, her desire to create a mathematical model for how the brain gives rise to thoughts and nerves to feelings, though she never achieved this. Her interest in the human brain came from her preoccupation inherited from her mother about her potential madness.

\vspace{0.3cm}

Ada first met Babbage in June 1833, through their mutual friend Mary Somerville. Shortly after, he invited her to see his prototype for his difference engine \cite{Toole}. She became fascinated by the machine and visited Babbage as often as she could so they could colaborate. Babbage was impressed by her intelectual and analytical skills, calling her "The Enchantress of Number". In 1843 he wrote to her: \cite{Enchantress}
\begin{quote}
    \textit{
Forget this world and all its troubles and if possible its multitudinous Charlatans—every thing in short but the Enchantress of Number.
}
\end{quote}

\vspace{0.3cm}

During a nine month period, Lovelace translated Luigi Manbrea's article on Babbage's newest proposed machine, the analytical engine, to which she appended a set of notes \cite{Sketch}. Explaining the analytical engine was a very difficult task as even a lot of other established scientists could not grasp the concept and the British establishment was uninterested in it. She even had to explain how the analytical engine differed from the original difference engine. Her work was very well, with Michael Faraday himself claiming to be a supporter of her writings.

\vspace{0.3cm}

The set of notes was three times longer than the article itself and included, in complete detail, a method of calculating a sequence of Bernoulli numbers which would have worked correctly on Babbage's analytical engine, had it ever been built. Based on this, she is now widely considered to be the first computer programmer and her method is recognised to be the first computer program ever. The notes also contain Lovelaces's dismissal of artificial intelligence:
\begin{quote}
    \textit{
        The Analytical Engine has no pretensions whatever to originate anything. It can do whatever we know how to order it to perform. It can follow analysis; but it has no power of anticipating any analytical relations or truths.
    }
\end{quote}
This claim has been the subject of a lot of debate, most notably what Alan Turing mentions in his paper: "Computing Machinery and Intelligence" \cite{AlanTuring}. He says that any machine that passes the Turing test can do \textit{anything}.

\newpage

\subsection{Potential of computing devices}

In her notes, Lovelace emphasised the difference between the analytical engine and the other calculating machines, particularly its ability to solve problems of any kind of complexity, realising that the machine had implications beyond number crunching: \cite{Sketch}
\begin{quote}
    \textit{
        [The Analytical Engine] might act upon other things besides number, were objects found whose mutual fundamental relations could be expressed by those of the abstract science of operations, and which should be also susceptible of adaptations to the action of the operating notation and mechanism of the engine...Supposing, for instance, that the fundamental relations of pitched sounds in the science of harmony and of musical composition were susceptible of such expression and adaptations, the engine might compose elaborate and scientific pieces of music of any degree of complexity or extent.
    }
\end{quote}

This analysis was a complete game changer since it reevaluated all other ideas about computing machines back then and it anticipated the usefulness of computers about one hundred years before they were even realised. According to the historian of computing and Babbage specialist Doron Swade:
\begin{quote}
    \textit{
        Ada saw something that Babbage in some sense failed to see. In Babbage's world his engines were bound by number...What Lovelace saw—what Ada Byron saw—was that number could represent entities other than quantity. So once you had a machine for manipulating numbers, if those numbers represented other things, letters, musical notes, then the machine could manipulate symbols of which number was one instance, according to rules. It is this fundamental transition from a machine which is a number cruncher to a machine for manipulating symbols according to rules that is the fundamental transition from calculation to computation—to general-purpose computation—and looking back from the present high ground of modern computing, if we are looking and sifting history for that transition, then that transition was made explicitly by Ada in that 1843 paper.
    }
\end{quote}

\newpage

\subsection{Controversy over her actual contribution}

Though she is considered the first computer programmer, some computer scientists and historians claim otherwise. One argument is that all of the programs in her notes had already been created by Babbage from three to seven years earlier. Ada discovered a 'bug' \cite{Bug} in it but there is no evidence that she ever prepared a program for the Analytical engine. Moreover, it is said that her correspondence with Babbage shows that she did not have the knowledge to do so \cite{AllanBromley}.

\vspace{0.3cm}

Bruce Collier wrote in his 1970 Harvard University PhD thesis that Lovelace "made a considerable contribution to publicizing the Analytical Engine, but there is no evidence that she advanced the design or theory of it in any way". Eugene Eric Kim and Betty Alexandra Toole consider it incorrect to say that Ada Lovelace was the first computer programmer in ther world since Babbage wrote the initial programs for the analytical engine, although the majority of them were never published. Bromley notes several dozen programs prepared by Babbage for the analytical engine that substantially predate Ada's notes.  Dorothy K. Stein regards Lovelace's notes as "more a reflection of the mathematical uncertainty of the author, the political purposes of the inventor, and, above all, of the social and cultural context in which it was written, than a blueprint for a scientific development".

\vspace{0.3cm}

On the other hand, Stephen Wolfram, in his book Idea Makers defends Lovelace's contributions. He admits that Babbage wrote several programs for his analytical engine that were never published but also mentions that "there's nothing as sophisticated—or as clean—as Ada's computation of the Bernoulli numbers. Babbage certainly helped and commented on Ada's work, but she was definitely the driver of it". He notes that her main achievement was to distill from Babbage's correspondence "a clear exposition of the abstract operation of the machine", something he never did.

\vspace{0.3cm}

Doron Swade, known for his work on Babbage, analyzed four claims about Ada during a lecture on Babbage's analytical engine:
\begin{enumerate}
        \item She was a mathematical genius
        \item She made an influential contribution to the analytical engine
        \item She was the first computer programmer
        \item She was a prophet of the computer age
\end{enumerate}
According to him, the fourth claim is of the most significance. He said that Ada was just a promising beginner, not a genius in mathematics and that she began to study basic concepts of mathematics five years after Babbage started working on his Analytical Engine so she could not have made serious contributions to the project. But he agrees that Ada was the only human at the time to see what the machine was capable of.

\newpage

\section{Commemoration}

The Ada programming language \cite{AdaLang} was first created in 1980 on the behalf of the United States Department of Defense and is targetted at real-time systems. Although not very popular today, the programming language is still being used in many important domains, for example in commercial aviation(Both Boeing and Airbus use Ada) and commercial rockets. The Department of Defense Military Standard for the language, MIL-STD-1815, was given the number of the year of her birth.

\vspace{0.3cm}

There is also a Lovelace Medal which is given to the winner of a competition specifically created for women students initiated by the British Computer Society. Lovelace Colloqium is an annual conference for women undergraduates and Ada College is a further-education college in Tottenham Hale, London, focused on digital skills \cite{AdaCollege}.

\vspace{0.3cm}

Ada Lovelace Day is an annual event held in Octomber, which began in 2009, which has the goal to "... raise the profile of women in science, technology, engineering, and maths" and to create new role models for girls and women in these fields. Events include Wikipedia edit-a-thons \cite{Editathon} with the aim of improving the representation of women on Wikipedia in terms of articles and editors to reduce gender bias on the website. At some point there was the Ada Initiative, a non profit organization dedicated to increasing the involvement of women in the free culture and open source movements \cite{OpenSource}.

\vspace{0.3cm}

Ada is also an inspiration and influence for the Ada Developers Academy in Seattle, Washington. The academy is a non-profit that seeks to increase diversity in tech by training women, trans and non-binary people to be software engineers.

\newpage

\section{Conclusion}

Technology is evil only when not used correctly, the same thing applies to pretty much everything else. No matter how much you may hate or like technology I think it is sensible to pay respect to people that try to make our lives easier. Even though Ada did not actually create a computer by herself she definitely saw the potential for it to help people in many areas like medical research. She conceptualized a lot of things that we now take for granted. All of this happened while computers did not even exist! You do not see such a visionary everyday.

\vspace{0.3cm}

There may be some people that doubt the importance of Ada Lovelace in the computing industry, but one thing is certain: she was a genius that first discovered the potential of computers and first published a working algorithm for them without actually having a computer. No other human can claim this!

\newpage

\printbibliography[title={Bibliography}]

\end{document}

